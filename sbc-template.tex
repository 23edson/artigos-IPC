% !TeX spellcheck = en_US
\documentclass[12pt]{article}

\usepackage{sbc-template}

\usepackage{graphicx,url}

\usepackage[brazil]{babel}   
%\usepackage[latin1]{inputenc}  
\usepackage[utf8]{inputenc}  
% UTF-8 encoding is recommended by ShareLaTex

     
\sloppy

\title{Ética na Area Cientí­fica}

\author{Edson Lemes da Silva, Lucas Cezar Parnoff}


\address{Universidade Federal da Fronteira Sul (UFFS) \\Chapeco -- SC -- Brazil
}

\begin{document} 

\maketitle

\begin{abstract}
  now
\end{abstract}
     
\begin{resumo} 
  
\end{resumo}


\section{Introdução}\label{sec:introducao}
Este artigo foi criado, para ter avaliação,
no curso de "Iniciação da Pesquisa Científica",
em que sua terceira tarefa é produzir um artigo
sobre "Ética na Pesquisa Científica".
\section{Metodologia} \label{sec:desenvolv}
O método utilizado neste artigo foi, realizar uma 
revisão bibliografica de artigos e sites, 
para desenvolver o mesmo, assim abrindo novas portas
para nos estudantes, assim elaborando artigos e
também, discuções sobre assuntos que afetam a 
sociedade científica.  
\section{Resultados e Discussão}\label{sec:resuldisc}
A palavra "ética" veio da palavra em grego "ethos", que
significa, "modo de ser" ou "caráter"\cite{signi:etmo}.
Com esse conceito podemos afirmar que a personalidade de
cada pessoa, é a ética que carrega consigo.Na area
científica, os principais conflitos que ocorrem, são
de origem, de alguma forma de falta de ética, como por 
exemplo, a falta de alguma refêrencia importante, nesse 
caso é feito uma citação em que não há referencias nem 
mesmo como conseguiu chegar ao resultado, este é mais 
conhecido como plagio.
A pesquisa científica foi desenvolvida para compartilhar 
conhecimentos e resultados, assim avançando a ciência 
em geral, sendo que foi mais obvio no crescimento 
populacional, pelo o aumento de 169 milhões de pessoas 
em 2000\cite{censo:00} para 190 milhões em 2010\cite{censo:10}.
Assim, enquanto a sociedade vai se desenvolvendo,
as pessoas vão recebendo, cada vez mais informações,
além de cada vez está mais dificil de guiar as
pessoas, para que não faltem com ética sobre 
outras pessoas, que provavelmente nunca a conheceram,
embora outros cientístas possam conhece-lo e utilizar
o trabalho que você fez, mesmo que tenha
feito uso de alguma negligência com algum outro 
pesquizador, desse modo, cada vez mais haverá
negligencias, consequentemente os
trabalhos precisaram ser muito bem avaliados
antes de alguma publicação, assim reduzindo
o número de publicações importantes para 
sociedade.




 

\bibliographystyle{sbc}
\bibliography{sbc-template}

\end{document}