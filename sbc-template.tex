% !TeX spellcheck = en_US
\documentclass[12pt]{article}

\usepackage{sbc-template}

\usepackage{graphicx,url}

\usepackage[brazil]{babel}   
%\usepackage[latin1]{inputenc}  
\usepackage[utf8]{inputenc}  
% UTF-8 encoding is recommended by ShareLaTex

     
\sloppy

\title{Ética na Area Cientí­fica}

\author{Edson Lemes da Silva, Lucas Cezar Parnoff}


\address{Universidade Federal da Fronteira Sul (UFFS) \\Chapeco -- SC -- Brazil
}

\begin{document} 

\maketitle

\begin{abstract}
  
\end{abstract}
     
\begin{resumo} 
  
\end{resumo}


\section{Introdução}\label{sec:introducao}
A palavra "ética" veio da palavra em grego "ethos", que significa, "modo de ser" ou "caráter"\cite{signi:etmo}. Com esse conceito podemos afirmar que a personalidade de cada pessoa, é a ética que carrega consigo.Na area científica, os principais conflitos que ocorrem, são
de origem, de alguma forma de falta de ética, como por exemplo, a falta de alguma refêrencia importante, nesse caso é feito uma citação em que não há referencias nem mesmo como conseguiu chegar ao resultado, este é mais conhecido como plagio.
\section{Conteúdo} \label{sec:conteudo}


\bibliographystyle{sbc}
\bibliography{sbc-template}

\end{document}