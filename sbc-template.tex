% !TeX spellcheck = en_US
\documentclass[12pt]{article}

\usepackage{sbc-template}

\usepackage{graphicx,url}

\usepackage[brazil]{babel}   
%\usepackage[latin1]{inputenc}  
\usepackage[utf8]{inputenc}  
% UTF-8 encoding is recommended by ShareLaTex

     
\sloppy

\title{Ética na Area Cientí­fica}

\author{Edson Lemes da Silva\inst{1}, Lucas Cezar Parnoff\inst{1}}


\address{Universidade Federal da Fronteira Sul (UFFS) \\Chapeco -- SC -- Brazil
}

\begin{document} 

\maketitle

\begin{abstract}
  now
\end{abstract}
     
\begin{resumo} 
  
\end{resumo}


\section{Introdução}\label{sec:introducao}

Este artigo foi criado, para ter avaliação,
no curso de "Iniciação da Pesquisa Científica",
em que sua terceira tarefa é produzir um artigo
sobre "Ética na Pesquisa Científica". 
realizar uma 
revisão bibliográfica de artigos e sites, 
para desenvolver o mesmo, assim abrindo novas portas
para nos estudantes, assim elaborando artigos e
também, discussões sobre assuntos que afetam a 
sociedade científica.  

\section{A ética e a moral}\label{sec:conteudo}


Primeiramente devemos apresentar o conceito de ética,sua origem e mostrar que é importante demonstra-lá em inúmeras ocasiões.

A ética está presente no mundo há muito tempo. Aristóteles  (384 AC - 322 AC) já discutia essses princípios por volta de 340 AC. Segundo ele "A maior virtude ética é a justiça". Sendo assim, Aristóteles acreditava que as regras morais nada mais eram que, segundo \cite{FABIO} "A vitória da razão sobre os fatos". Além dele, algun pensadores religiosos também mostraram seus pensamentos sobre o assunto. Tomás de Aquino (1225 - 1274) disse: "Todas as ações humanas devem ser dirigidas a um fim último, a felicidade de estar com Deus". Conforme \cite{ORLANDO} "A ética é vista como a aplicação dos princípios morais, segundo Tomás de Aquino". Outro religioso importante foi Spinoza ( 1632 - 1677), segundo ele : "Não há bem ou mal absoluto; as más ações são feitas por aqueles que não conhecem Deus". Ele aceita que a ética está relacionada com a aplicação dos conhecimentos. O sociólogo Max Weber (1864 - 1920) apresentou conceitos específicos, tais deles: Ética da convicção ("Conjunto de normas e valores que orientam o comportamento na sua esfera privada"), e Ética da responsabilidade ("Conjunto de normas e valores que orientam a decisão na vida pública"), ou seja, para Weber esses conceitos são realistas, são aplicações da razão pragmática. Muitos outros pensadores mostraram suas ideias sobre a ética, ainda hoje, existem estudiosos que visam o aperfeiçoamento do mesmo.

Podemos definir ética em poucas palavras: bom costume. O seu papel engloba a busca por hábitos que contribuam para o melhor convívio entre os seres humanos. A ética basicamente é um conjunto de ideias construídas a partir do comportamento humano em sociedade. Assim, se define regras morais em prol do convívio social.

Em sociedade, as pessoas aprendem a exercer a ética quando crianças,e levam essa ideia para toda a vida. Obviamente que na prática isso não acontece, existem pessoas contrárias aos princípios morais,e infelizmente é muito comum, exemplo disso : roubos e assassinatos. Estas pessoas são ditas antiéticas.

Outro conceito muito importante, é a moral. Nela estão definidas regras que devem ser aplicadas no cotidiano. O ser humano, deve saber identificar o que é certo e o que não é. O que é bom, e o que é ruim. Na prática a moral e a ética são semelhantes, já que as duas se referem ao comportamento humano.

A aplicação destes conceitos é muito importante em qualquer situação. Uma vez que, estão presentes em todo lugar, e em todo tipo de atividade. A ética é aplicável em inúmeras áreas, dentre elas, a ética empresarial, na pesquisa científica e profissional.


\subsection{Ética e a pesquisa científica} \label{sec:sub1}

Após definirmos o conceito de ética. Mostraremos a sua aplicação, e a relação com a pesquisa científica. Ela foi desenvolvida para compartilhar 
conhecimentos e resultados, assim avançando a ciência 
em geral, por exemplo, o crescimento 
populacional, pelo o aumento de 169 milhões de pessoas 
em 2000\cite{censo:00} para 190 milhões em 2010\cite{censo:10}.

Em primeiro lugar, devemos saber a importância da pesquisa científica. Afinal, é através dela que muitos problemas são resolvidos. Para se achar uma solução, basta fazer uma única tarefa: produzir conhecimento. Seja ele qual for. Buscar conhecimento, significa elaborar provas concretas, incluindo o óbvio, saberes culturais e práticos, levando em conta que, estes conhecimentos levem a um único objetivo : contribuir para o cotidiano.

Como dito anteriormente, a pesquisa científica só existe quando há um propósito para tal. Em outras palavras, o esforço em investigar algo para avançar o conhecimento, é apenas feito quando existe uma razão aceitável. Seja ela porque chamou a atenção, seja um problema ou por desafio. E quando existe uma pesquisa científica com um propósito definido, então automaticamente devemos assumir questões de boa conduta sobre o mesmo, ou seja, conforme \cite{PETER} "Ao embarcar em um trabalho científico, assumimos questões éticas e morais simultaneamente".

A ética na pesquisa científica parece ser bem óbvia, afinal ninguém deve ser antiético em pleno século XXI. Certamente que estes princípios morais não são recentes, como demostrado na sessão \ref{sec:conteudo}. Existem inúmeros exemplos de pesquisas científicas que podemos usar como antiéticas, dentre elas podemos citar: durante os períodos de guerras do século passado, onde foram realizados experimentos com cobaias humanas, injetando células anormais. Tudo isso para verificar o comportamento do organismo humano em prol da ciência. 

Certamente pesquisas como: estudo de doenças, estudos genéticos são bastante rígidas, e podem gerar inconveniência por parte da sociedade. Questão desse tipo devem ser cautelosas, o bastante para não serem contra os princípios morais.

Além disso, outras situações em que se aplicam os valores morais, dentre elas estão : biopirataria e o plágio. Usar algo sem as devidas referência é inaceitável, antimoral. Podemos pensar assim, com publicações de artigos, onde na cópia do mesmo não está definida a sua origem.







 

\bibliographystyle{sbc}
\bibliography{sbc-template}

\end{document}