% !TeX spellcheck = en_US
\documentclass[12pt]{article}

\usepackage{sbc-template}

\usepackage{graphicx,url}

\usepackage[brazil]{babel}   
%\usepackage[latin1]{inputenc}  
\usepackage[utf8]{inputenc}  
% UTF-8 encoding is recommended by ShareLaTex

     
\sloppy

\title{Ética na Area Científica}

\author{Edson Lemes da Silva\inst{1}, Lucas Cezar Parnoff\inst{1}  }

\address{Universidade Federal da Fronteira Sul
  (UFFS)\\
  Chapecó -- SC -- Brasil
\email{ed \textunderscore edsoon@hotmail.com, ice0life@gmail.com}
}

\begin{document} 

\maketitle

\begin{abstract}
  
\end{abstract}
     
\begin{resumo} 
  
\end{resumo}


\section{Introdução}

\section{A ética e a moral}\label{sec:conteudo}


Primeiramente devemos apresentar o conceito de ética, mostrar que é importante demonstra-lá em inúmeras ocasiões.
Podemos definir ética em poucas palavras: bom costume. O seu papel engloba a busca por hábitos que contribuam para o melhor convívio entre os seres humanos. A ética basicamente é um conjunto de ideias construídas a partir do comportamento humano em sociedade. Assim, se define regras morais em prol do convívio social.

Em sociedade, as pessoas aprendem a exercer a ética quando crianças,e levam essa ideia para toda a vida. Obviamente que na prática isso não acontece, existem pessoas contrárias aos princípios morais,e infelizmente é muito comum, exemplo disso : roubos e assassinato. Estas pessoas são ditas antiéticas.

Outro conceito muito importante, é a moral. Nela estão definidas regras que devem ser aplicadas no cotidiano. O ser humano, deve saber identificar o que é certo e o que não é. O que é bom, e o que é ruim. Na prática a mortal e a ética são semelhantes, já que as duas se referem ao comportamento humano.

A aplicação destes conceitos é muito importante em quaiquer situação. Uma vez que, estão presentes em todo lugar, e em todo tipo de atividade. A ética é aplicável em inúmeras áreas, dentre elas, ética empresarial, na pesquisa científica, profissional.


\subsection{Ética e a pesquisa científica} \label{sec:sub1}




\bibliographystyle{sbc}
\bibliography{sbc-template}

\end{document}
